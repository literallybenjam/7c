\chapter{Origins and Early Developments}

The year was 2009, and I had a dilemma.
I had just started work on a new idea for a novel—originally, \emph{The Seven Chronicles}; later, \emph{The Chronicle of the Seven}; shorthand, \emph{7c}—and, well, I needed to come up with some names.
It was an epic fantasy of global proportions, set in an imaginary world, and, well, it just didn't make sense to name everything in English.
But I wasn't sure how I felt about appropriating other languages and cultures for my own purposes, either; what I needed, I decided, were languages and cultures of my own invention, that I could use without fear of falling into negative stereotypes or offenses.

This is easier said than done, of course, but I had plenty of free time on my hands, so I figured I'd give it a shot.
Languages, I supposed, probably originated from a few core sounds, which were combined and modified in new and exciting ways, eventually resulting in the vast level of complexity and diversity that we see today.
(I have since revised this hypothesis—but what can I say? I was fifteen.)
I pulled a few words out of thin air—things like \emph{doi}, “go” and \emph{tah}, “that”—and set to work.

I worked on three languages, in total; of these, only one achieved any level of completion.
It was called \emph{Jästūgei}, and by the end of 2012 it was in its eighth iteration, with a (mostly) complete grammar and orthographical system.
It was a moraic language, OVS word-order if you can believe it, with a relatively small vocabulary but a unique character and feel.
Thus accomplished, I decided to take a break—\emph{7c} was no longer my highest priority, and I was in college, now.
The language sat dormant for a couple of years.

In that time, my understanding of language evolved and my priorities shifted.
In the summer of 2014, I began work on a new language with a radically different approach—\emph{Fizeng}, as it is now called.
I considered Jästūgei to be something of an experiment, one that I learned a lot from, but which ultimately was misguided and doomed to failure.

A year passed, and I decided it deserved another chance.

I took all three of the languages that I had created in the period from 2009–2012 and combined them into a single lexicon, deciding to use this as a foundation for a new language family, one which incorporated my evolved understanding to give my older attempts new meaning.
I named the language \emph{Sevensi}, after the project that had inspired its creation, and began development anew.

I recount this history not because it is overtly relevant to the subject matter of this volume—the development history of Proto-Jastu-Sevensi is, for the purposes of our work, pre-historic—but because it provides a sense of understanding for the varied approaches and understandings under which the language has grown and evolved over the years.
Almost everything I know about language I have learned \emph{while} working on this effort, and that means that my early attempts, misguided as they may have been, differed significantly from the current direction and emphasis I take to the project.
Sevensi encapsulates all of this in its foundations: Its root words and structures are formed out of youthful naïvety, its evolution has taken place through adventurous curiosity and experimentation; as a language, it has grown up with its inventor.
And while many elements of the language have been revised and made more mature, this trace is impossible to ever completely remove.

\section*{Situation and Context}

Imagine a continent, roughly square in shape, surrounded on all sides by ocean and featuring an extensive archipelago past its western border.
This is the setting in which the Sevensi language is said to develop.
It is a land of diverse geography, whose people find a living through both trade and agriculture, where nations and rulers have risen up, fragmented, and come to strength once again.
Cut off from the outside world, it has developed its own unique cultures and philosophies—far from monolithic—and embedded into its language is a unique way of viewing the world.

What we know of early Sevensi culture is limited: vernacular stories that have been passed down through the ages, subtle  references to past occurances in the Classics, remnants of older worldviews in the stories and legends of today.
Most written documents from this time were either lost or else written in a logographic script no longer understood, and so we find ourselves tracing backwards from the present, attempting to reconstruct a long-forgotten past.

