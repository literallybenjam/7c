\chapter{Origins and Early Developments}

The year was 2009, and I had a dilemma.
I had just started work on a new idea for a novel—originally, \emph{The Seven Chronicles}; later, \emph{The Chronicle of the Seven}; shorthand, \emph{7c}—and I needed to come up with some names.
It was an epic fantasy of global proportions, set in an imaginary world, and, well, it just didn't make sense to name everything in English.
But I wasn't sure how I felt about appropriating other languages and cultures for my own purposes, either; what I needed, I decided, were languages and cultures of my own invention, that I could use without fear of falling into negative stereotypes or offenses.

This is easier said than done, of course, but I had plenty of free time on my hands, so I figured I'd give it a shot.
Languages, I supposed, probably originated from a few core sounds, which were combined and modified in new and exciting ways, until they eventually produced the vast level of complexity and diversity that we see today.
(This hypothesis has since been revised—but what can I say? I was fifteen.)
I pulled a few words out of thin air—things like \emph{doi}, “go” and \emph{tah}, “that”—and set to work.

I worked on three languages, in total; of these, only one achieved any level of completion.
It was called \emph{Jästūgei}, and by the end of 2012 it was in its eighth iteration, with a (mostly) complete grammar and orthographical system.
It was a moraic language, OVS word-order if you can believe it, with a relatively small vocabulary but a unique character and feel.
Thus accomplished, I decided to take a break—I was in college, now, and \emph{7c} was no longer my highest priority.
The language sat dormant for a couple of years.

In that time, my understanding of language evolved and my priorities shifted.
In the summer of 2014, I began work on a new language with a radically different approach—\emph{\han{斤丸舌} [Fizeng]}, as it is now called.
I considered Jästūgei to be something of an experiment, one that I learned a lot from, but which ultimately was misguided and doomed to failure.

A year passed, and I decided it deserved another chance.

I took all three of the languages that I had created in the period from 2009–2012 and combined them into a single lexicon, deciding to use this as a foundation for a new language family, one which incorporated my evolved understanding to give my older attempts new meaning.
I named the language \emph{Sevensi}, after the project that had inspired its creation, and began development anew.

I recount this history not because it is overtly relevant to the subject matter of this volume—the development history of Proto-Jastu-Sevensi is, for the purposes of our work, pre-historic—but because it provides a sense of understanding for the varied approaches and understandings under which the language has grown and evolved over the years.
Almost everything I know about language I have learned \emph{while} working on this effort, and that means that my early attempts, misguided as they may have been, differed significantly from the current direction and emphasis I take to the project.
Sevensi encapsulates all of this in its foundations: Its root words and structures are formed out of youthful naïvety, its evolution has taken place through adventurous curiosity and experimentation; as a language, it has grown up with its developer.
And while many elements of the language have been revised and made more mature, this trace is impossible to ever completely remove.

\section{Situation and Context}

Imagine a continent, roughly square in shape, surrounded on all sides by ocean and featuring an extensive archipelago past its western border.
This is the setting in which the Sevensi language is said to develop.
It is a land of diverse geography, whose people find a living through both trade and agriculture, where nations and rulers have risen up, fragmented, and come to strength once again.
Cut off from the outside world, it has developed its own unique cultures and philosophies (far from monolithic), and embedded into its language is a unique way of viewing the world.

What we know of early Sevensi culture is limited: vernacular stories that have been passed down through the ages, subtle  references to past occurances in the Classics, remnants of older worldviews in the stories and legends of today.
Most written documents from this time were either lost or else written in a logographic script no longer understood, and so we find ourselves tracing backwards from the present, attempting to reconstruct a long-forgotten past.

Although it is, in some ways, dangerous to attempt to draw too many paralells between the Sevensian Continent and other locations in or beyond this world, such a practice can perhaps shed some light on the conditions in which early Sevensi culture might have been said to form.
We can suppose, for example, that early Sevensian humans first began cultivating crops as a food source about 10,000 years ago, alongside the rest of the world.
From linguistic data and basic geographical understanding, we can conclude that rice production was likely common in the Southern regions; wheat cultivation is highly probable elsewhere.
A turn to agriculture led to more fixed settlements and the development of new tools and storage materials—we can see pottery and stone polishing as emerging in this period as well.

The Jastu-Sevensi languages are influenced by three distinct etymological lines, and in these we can see at least three major prehistoric Sevensi cultures.
The most principal of these, \textbf{Jāstulæ} /dʒeɪstuːlej/, probably inhabited a central region on the continent, near a river to facilitate wheat agriculture, and expansive enough to be in some contact with the neighbouring cultures.
To the north, \textbf{Lrex} /ɛlɹɛks/ culture thrived on a combination of farming and herding; to the south, \textbf{Eho} /iːhəʊ/ culture cultivated rice, gathered freshwater plants, and were heavily involved in fishing. And within each of these, numerous sub-cultures doubtlessly developed and thrived.

Our knowledge of these early cultures is highly limited, largely because of the lack of any records dating back to their existence.
We know that an early form of Sevensi logographic writing first developed within Jāstulæ culture, and some record of its existence is chronicled in texts as early as the Classical period, but little remains to analyse today.
The situation is worse for the cultures of Lrex and Eho; little of these languages has survived in the Jastu-Sevensi language family, and there is no evidence that either culture had developed any form of writing.
Without direct archeological access to the Sevensi Continent, there is little we can learn beyond the bits and pieces that might be reconstructed through reference in early Sevensi historical works.

\section{The Archaic Period}

In order to properly understand Classical Sevensi, both as a language and as a culture, it is important to know what preceeded it.
This period, stretching from the invention of Jāstulæ writing until the development of the earliest recognizable form of what we might classify as “Classical,” is largely shrouded in mystery, in no small part because hardly any linguistic evidence from the period exists.
In fact, to date only one text from the period has received a full translation, reproduced alongside the original below (Figure \ref{archaic-translation}).

\begin{figure}[ht]

\centering\small

\begin{tabular}{p{200pt}p{200pt}}
{\raggedleft Ʒā pœt zafo ʒā dätzē dotl’go.
Pǝ dāso zafo hämātl’go pǝ dätzē.
Ʒā ħœfo dohil'go su.
Ʒā tuād ʒoħ jäsātol’gi tusol’go su.
Sat äħꜵ pǝ dāso tutā pǝ dätzē, mœ pǝ hoʒā ʒoħ dosil’go su.\par}
&{\raggedright Once upon a time (lit., \emph{in a moment}), there was a fox.
The fox [lived?] in the water.
A sadness was had by him.
He wanted somebody to speak to.
Because of this, the fox departed the water, and he came to the land.\par}
\\{\raggedleft “Pə dätzē mœ pə fäʒēl”\par}
&{\raggedright “The fox and the crabs”\par}
\\{\raggedleft Pə dätzē āʒo hoʒā ʒoħ ʒoħʒəl'ħə iə ʒā fäʒēl vodə howosēl'go su.
Jäsātol'go su, “Odo!”, o su ʒoħ fil'jäsātol'go pə fäʒēl.
Jäsātol'go su, “Jäsātol'gi howāl əwādl?”.
Su ʒoħ fil'jäsātol'go sul.
Zo ʒā so fil'dotl'go howāl'ħə howosēl'go pə dätzē; sat äħꜵ totāl'go su.\par}
&{\raggedright The fox arriving onto The Land, he saw some crabs.
Said he, “Hello!”, but the crabs did not speak to him.
Said he, “Do you speak?”
They did not speak to him.
The fox saw that he wasn't making progress; because of this, he departed.\par}
\end{tabular}

\caption{\emph{Pə dätzē āʒo dāʒo}, an ancient Jastu-Sevensi text.}
\label{archaic-translation}

\medskip\hrule

\end{figure}

The significance of this text is not clear.
It appears to date from very early in the Archaic period, evident from the use of the suffix \emph{-l} to denote pluralization, a feature which was later dropped from the language.
That the text is incomplete is apparent; in all likelihood, the text was intended to convey an abstract truth or moral value, but from this small fragment it is impossible to tell what such a message might be.

We find in this text several other features indicative of Archaic-period Jastu-Sevensi languages: an object-verb-subject word order, the use of \emph{l'} to introduce affixes; a few others, such as tense markings, are present only in particular time periods and regions.
The poetic nature of the language is somewhat limited; the sentences are simple, the vocabulary basic.
Nevertheless, the text is able to convey a narrative, clearly invented, with relative ease, perhaps speaking to the extent that creative storytelling may have influenced Sevensi culture at this early stage.

It is important to remember that text excerpts such as these were not originally composed in Latin characters.
The transcription above employs the system of romanization in use during this stage of language development, but the original composition employed logographic characters whose pronunciation is rarely visually apparent.
While some words have clear descendants in more modern forms of the language, others do not, and the phonologies hinted at above are entirely speculative in nature.

\section{The Pre-Classical and Early Classical Periods}

Broadly speaking, the term Classical Sevensi can be used to refer to any early form of Sevensi after its split from the broad Jastu-Sevensi family; in particular, it denotes any iteration of language after the development of three-way voiced/voiceless/aspirated distinction and rigid vowel harmonization structures but prior to the grammatical introductions of Middle Sevensi.
Colloquially, however, Classical Sevensi generally refers to the language as it is spoken in the Late Classical period—that is, the period in which the great Sevensi Classics were written.
This period is preceded by two earlier ones: the Pre-Classical and Early Classical periods.

The Pre-Classical period is the period in which many significant aspects of what we now consider to be Classical Sevensi begin to take shape.
It is the period in which Sevensi literature first begins to proliferate, and many poems and documents included in the later Classics are first composed during this time.
Many of the grammatical structures found in Classical Sevensi are lacking or come in a markedly different form, but we see a vast lexical expansion and far better-established modes of word-formation emerge.

The Early Classical period sees the establishment of many of the formalized grammatical structures that come to characterize Classical Sevensi, including its complex system of verb conjugation.
We see the scope of literature expand in this period and formalized systems of poetry become better-defined.
