\frontmatter
\pagestyle{empty}
\openright

%%  HALF-TITLE  %%

\begin{vplace}[0.25]\begin{center}
\begin{large}CLASSICAL SEVENSI\end{large}\\
History and Development
\end{center}\end{vplace}

%%  TITLE AND COPYRIGHT  %%

\cleardoublepage

\noindent\begin{huge}CLASSICAL SEVENSI\end{huge}\\
\begin{large}\emph{History and Development}\end{large}

\medskip

\noindent BENJAMIN SHOEMAKE

\newpage

\begin{small}

\setlength{\parindent}{0pt}
\setlength{\parskip}{\medskipamount}

This work has been placed by its author in the public domain under a Creative Commons CC0 1.0 Universal License.
You can find out more about this license, and read the full legal text, at \urireftext{http://creativecommons.org/publicdomain/zero/1.0/}{http://creativecommons.org/publicdomain/zero/}\linebreak[0]\urireftext{http://creativecommons.org/publicdomain/zero/1.0/}{1.0/}.

The source code for this work is available at the official Sevensi git repository, \uriref{https://github.com/literallybenjam/7c}.
Its author may be reached there or via \urireftext{https://twitter.com/literallybenjam}{@literallybenjam} on Twitter.

This work is formatted via \XeLaTeX. Body text is set at 17pt. The following font families are used:

\begin{tabular}{@{\hspace{2em}}l@{\hspace{1em}}l}
Junicode&English text\\
\hangul{Nanum Myeongjo}&Hangul text\\
\hangul{\emph{Nanum Brush Script}}&Hangul italics\\
\han{Songti SC}&Hanzi text\\
\han{\emph{Libian SC}}&Hanzi italics\\
\end{tabular}

\noindent Working Draft published \today.

\end{small}

%%  CONTENTS  %%

\chapter{Contents}
\pagestyle{simple}
\begin{table}[hbp]

\begin{small}

Introduction / \pageref{introduction}

\medskip

A Note on Orthography / \pageref{orthography}

\end{small}

\bigskip

\textbf{Classical Sevensi: History and Development / \pageref{text}}

\end{table}

%%  INTRODUCTION  %%

\chapter{Introduction} \label{introduction}

It might seem odd, by some standards, at first: An extensive history of a newly-invented language?
How can such a thing be possible?
And yet: All languages change and evolve over time, and invented ones are no exception.
This book outlines the social, cultural, linguistic, and narrative history of Classical Sevensi and its precursors, insofar as such a thing can be said to exist.

Classical Sevensi poses a unique problem for this sort of project in that its development has taken place over the course of years, not centuries.
In some ways, this makes the narrative simpler, more accessible to historians, on more stable ground.
But there is an aspect of language that resists such an easy explination.
Just as language shapes narratives, narratives shape language, and a reduction of Sevensi's unique cultural identity to the story of one person's adventurous exploits into the linguistic structure threatens to eradicate any meaning or significance that the Classical phase might hold.
If the history of Sevensi was merely the story of an invented language's construction, what would Classical Sevensi be but as merely a stage in that proceedure, a beta version to the eventual final release?

And so we find ourselves not only the recorders, but also the creators, of history.
Because Classical Sevensi \emph{is} important, it holds unique cultural and linguistic relevance to any study of the language as a whole, and it has a unique character that is worthy of recognition.
Histories are, at their truest form, about narrative, the construction of a story to help explain and describe the many events and stages which have allowed us to acheive our current situation, whatever that may be.
And, as we will find, constructing such a narrative around Classical Sevensi is essential to our understanding of its significance in relation to the language in its totality, and to \emph{language} at-large.

This is not to say we should ignore the very real circumstances of Sevensi's real-world development.
To the contrary.
Rather, it is to say that a cultural and historic understanding of the language must come from multiple sources, from our outside understanding, but also from the stories that the language tells itself in its evolutions and structures.
Like the language itself, some aspects of this history will be constructed, invented.
But they will also be present, in a very real form, within the language's structure, within its vocabularies and modalities, its etymological foundations and its subtle connotations.
“Real” or not, they are essential to any study of Classical Sevensi in its own right, on its own ground.
And while such a history, like any history, will necessarily be incomplete, this book aims to take that first step.

\chapter{A Note on Orthography} \label{orthography}

For much of its history, Classical Sevensi was written using logographic characters, many of which doubled as morphological and phonological markers.
These characters are described to some extent in this \emph{History}, but their forms have largely been lost to history.

Near the end of the Classical period, a Sevensi alphabet was invented, and it is these spellings which are reproduced here.
This text follows the modern tradition of using Korean Hangul in place of traditional Sevensi characters; although the forms might vary somewhat from a true Sevensi alphabet, their meanings are preserved.
The presence of Hangul in the language is a point of ahistoricality—nowhere does Sevensi culture find itself in contact with that of Korea—but this contradiction should not be seen as threatening.
To the contrary: It perfectly epitomises the compromise between narrative and reality that the language itself is build upon.

With the understanding that not all readers will necessarily be familiar with Hangul, parenthetical romanizations are provided alongside native spellings.
The manner of romanization is that used during real-world language development; it stresses phonological and etymological clarity over readability.
However, because this is a study of a Classical language, note that the romanizations and pronunciations provided here make no claim to complete accuracy; the way that the language was “really spoken” is a matter of speculation.
